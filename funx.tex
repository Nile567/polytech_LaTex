\documentclass[12pt]{article}
\usepackage[utf8]{inputenc}
\usepackage[russian]{babel}
\usepackage{amsmath}
\usepackage{hyperref}
\numberwithin{equation}{section} 
\begin{document}
	 \tableofcontents 
	\newpage
	\section{Функции и их классы эквивалентности}
	Пусть $D$ и $R$ -- конечные множества, $K$ -- коммутативное кольцо, $\omega$ : $R \rightarrow K$ -- весовая функция на множестве $R$. Для каждой функции $f$ из множества $\mathcal{F} = {f : D \rightarrow R}$ определим ее вес $ \omega $ положив $\omega(f)= \prod_{d\in D} \omega\left( f(d)\right)  $.Функции $f_{1}$ и $f_{2}$ назовем эквивалентными, если найдется такой элемент $g$ группы $G$, что $f_{1}(d) = f_{2}(gd)$ для каждого $d \in D.$ Очевидно, что множество $\mathcal{F}$ распадается на классы эквивалентности $F_{1},\ldots,F_{k}$, и так как веса эквивалентных функций из одного класса совпадают, то можно говорить о весе класса эквивалентности функций из $\mathcal{F}$. Вес класса $F$ обозначим через $W(F)$.
	
	Введенные определения снова рассмотрим на примере задачи о раскраске граней кубика. Грани будем раскрашивать в два цвета — черный и белый. В этом случае множество $D$ состоит из шести граней кубика, а множество $R$ -- из черного и белого цветов. Кубик, грани которого покрашены в черный и белый цвета, будем рассматривать как функцию из $D$ в $R$, которая ставит в соответствие каждой грани ее цвет. Группой $G$, действующей на множестве $D$, будет рассмотренная выше группа вращений кубика, а две функции будут эквивалентными, если соответствующие им раскрашенные кубики можно преобразовать друг в друга при помощи вращений из группы $G$. Например, нетрудно видеть, что все кубики с одной черной и пятью белыми гранями эквивалентны друг другу. В качестве кольца K возьмем кольцо многочленов от переменных $x$ и $y$ с целыми коэффициентами, при этом белому цвету припишем вес $x$, а черному-- $y$. Таким образом, весом раскрашенного кубика, и весом соответствующей ему функции, будет одночлен шестой степени от переменных $x$ и $y$. Если нас интересует число различных кубиков с тремя черными и тремя белыми гранями, нам надо найти число классов эквивалентности, вес которых равен $x_{3}y_{3}$. Сделать это можно при помощи теоремы Пойа.
\end{document}